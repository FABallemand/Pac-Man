\section{Conclusion}
The resulting program is a good looking and very playable Pac-Man video game. The main elements from the iconic video game are present: Pac-Man, ghosts, dots, energizers, fruits... and work as one would expect. Retro gamers will definetly notice it is not exactly the original game from Samco but this project should be considered as a proof of concept rather than a real game as values (like speeds, timers, scores...) and ghost strategies has not been fine tuned to deliver Pac-Man's well known challenging experience. Nor do they change according to levels. However, all the elements required to recreate the full experience are implemented.

Learning advanced techniques of C++ programming while creating a complex program is not an easy task. Basic C++ features like classes and heritage are massively used and we tried to take advantage of many modern C++ features (iterators, functor, lambda function...). Of course, more experienced programmers will find flaws in the program. To name but one, multiple heritage should have been used in order to split object competencies. For example, ghosts are both moveable and eatable objects. But because we worked in a timely constrained environnement, we had to focus on developping a working program rather than creating a video game engine from scratch.

Working on this project was very challenging but also quite enjoyable as we learned advanced C++ programming in an interesting and funny way.