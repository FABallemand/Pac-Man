\section{Project Management}
This project requires many hours of work to obtain a game that is both good looking and playable. In order to work efficiently, we set up some rules and define project organisation.

We decided to use a GitHub repository in order to share files and keep track of versions. GitHub also propose project management features like \textit{issues} that are really usefull when it comes to assign tasks to a member of the team.

Most of the time we worked together on the same computer or via a VSCode \textit{Live Share} as it allows us to think, discuss and program with less errors as one of us is typing while the other is supervising. However, we sometimes distributed the tasks in order to work in parallel and speed up the developement. In this case, we made sure to read the modifications from the other as soon as possible using the commit history on GitHub.

From time to time, we organised ten-minutes-long meetings on a white board in order to plan and start thinking about future tasks.

We usually used Discord in order to communicate via text messages to share ideas or information outside of meetings or via calls during the Live Share sessions.