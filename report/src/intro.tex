\section{Introduction}

\begin{quote}
    Pac-Man, originally called Puck Man in Japan, is a 1980 maze action video game developed and released by Namco for arcades. In North America, the game was released by Midway Manufacturing as part of its licensing agreement with Namco America. The player controls Pac-Man, who must eat all the dots inside an enclosed maze while avoiding four colored ghosts. Eating large flashing dots called "Power Pellets" causes the ghosts to temporarily turn blue, allowing Pac-Man to eat them for bonus points.\cite{pacman_wiki}
\end{quote}

\begin{quote}
    Object-oriented programming (OOP) is a programming paradigm based on the concept of "objects", which can contain data and code. The data is in the form of fields (often known as attributes or properties), and the code is in the form of procedures (often known as methods).\cite{oop_wiki}
\end{quote}

The goal of this project is to recreate the Pac-Man video game from 1980 as accuratly as possible using C++.
C++ is a object-oriented programming language that is very efficient and regulraly updated. In this project, many features from the latest C++ standards (C++11/14/17/20) were used.
The Standar Template Library (STL)\cite{stl_wiki} and the Simple DirectMedia Layer (SDL)\cite{sdl} add even more features to the basic C++ language.

GitHub: \url{https://github.com/FABallemand/Pac-Man}